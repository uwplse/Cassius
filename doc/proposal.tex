\documentclass{simple}

\title{Project Proposal: Synthesizing CSS from Mock-ups}
\author{Pavel Panchekha}

\begin{document}
\maketitle

Web pages are often designed in a visual image editor,
  and only later laid out as HTML styled by CSS.
The process of converting such a \emph{mock-up}
  into an HTML document and a CSS style file
  can be complex,
  due to the subtleties of CSS as a layout language.
Even when the task is straight-forward,
  it requires skill in using CSS,
  which many designers do not have.
I propose to automate this process:
  to produce CSS and HTML automatically from mock-ups.

\paragraph{Mock-ups}

The input to my tool should be a collection of mock-ups,
  each of which is a image of how a web-page should be rendered
  and fragments of the HTML documents.
The fragments are text and images
  that are not drawn by the browser.
The specification language is thus an image and HTML fragments,
  not a traditional language.

\paragraph{Output}

The tool will produce
  a complete HTML document for each mock-up
  (this document built out of the HTML fragments in that mock-up,
  and other markup to connect those fragments into a full HTML document)
  and a single CSS style file.
Rendering each resulting document according to this style file
  should correspond, as closely as possible, to the mock-up image.
The output language is thus CSS and HTML.
(Which exact CSS standard to use is to be determined,
  potentially by the user.)

\paragraph{General approach and project plan}

One difficult part of CSS layouts is the box model.
I will attempt to encode the box model as a linear programming problem
  and map sparse solutions to CSS rules.
Starting with a simple box model (say, no positioning rules)
  will simplify the task; I'll add complications iteratively.
If the input language is images, some computer vision will be necessary
  to identify various elements of the page.
This can be avoided at first by manually identifying boxes
  that the synthesizer then arranges.
More complex CSS features (floats, positioning, \textsf{@media} queries)
  may require complex decision-making,
  which an SMT solver may be useful for.

\paragraph{Evaluation}

Several sources of benchmarks exist.
The first is CSS tutorials,
  such as those on W3Schools and the Mozilla Developer Network.
The second is articles in industry journals such as A List Apart,
  which often describe how to achieve particular visual effects
  using CSS markup
  (often CSS markup restricted a to subset
  that is interpreted similarly by different browsers).
The third is downloading the main pages of major websites
  (as ranked by, for example, Alexa)
  or common blog templates (from Blogger, Wordpress, and similar)
  and attempting to recreate the CSS from scratch.
In each case, the objective is the same:
  to re-create the desired visual effect.

It would be good to do a survey of some designers as well---%
  if my claim is that designers don't really like this step,
  it should be checked.
The same designers could be asked to evaluate the generated CSS,
  to ensure that the result is maintainable.
  
% Response:
%
% I like the idea, and 599 is definitely the right place to pursue it.
% But we should talk about making it more PL-y if you want to work on
% this beyond 599. My main concern is this: it may be hard to get a PL
% audience (e.g., at PLDI) interested in the problem. Of course, we
% could target HCI conferences instead, although that’s a gamble
% without having a person on the team who speaks HCI.
  
% Some twists to consider adding to the problem: 

% Synthesis of JavaScript code to add functionality to web pages. What
% kinds of programs would be interesting? Who would we target?
% Experienced developers, or novices, or end users?

% Synthesis of certain kinds of web-apps from higher level specs. For
% example, something like this but for a different class of apps.

% Synthesis of mobile apps?



\end{document}